\documentclass[11pt,xcolor={table}]{beamer}
\usetheme{unicauca}
\usepackage[utf8]{inputenc}
\usepackage[spanish]{babel}
\usepackage{amsmath}
\usepackage{amsfonts}
\usepackage{amssymb}
\usepackage{graphicx}
\usepackage{pgfplots}
\usepackage{colortbl}

\author{Jhon Jairo Alvarez Londoño}
\title{Mi exposición en Latex}
%\setbeamercovered{transparent} 
%\setbeamertemplate{navigation symbols}{} 
\logo{\includegraphics[scale=0.1]{logo-unicauca.png}} 
\institute{Universidad del cauca} 
%\date{} 
\subtitle{Mi subtitulo}

%Volviendo a configurar la página de los titulos 
%para cambiar sus posiciones respecto a la nueva 
%imagen de fondo.
\setbeamertemplate{title page}
{
  \vbox{}
  \begingroup
    \centering
    \vskip-55pt
    \begin{beamercolorbox}[rounded=true,sep=5pt,center]{title}
      \usebeamerfont{title}\inserttitle\par%
      \ifx\insertsubtitle\@empty%
      \else%
        \vskip0.25em%
        {\usebeamerfont{subtitle}\usebeamercolor[fg]{subtitle}\insertsubtitle\par}%
      \fi%
    \end{beamercolorbox}%
    %\vskip1em\par
    \vskip-3pt
    \begin{beamercolorbox}[sep=8pt]{author}
      \usebeamerfont{author}{\color{white}{\insertauthor}}
    \end{beamercolorbox}
    \vskip-12pt
    \begin{beamercolorbox}[sep=8pt]{institute}
      \usebeamerfont{institute}{\color{white}{\insertinstitute}}
    \end{beamercolorbox}
    \vskip-12pt
    \begin{beamercolorbox}[sep=8pt]{date}
      \usebeamerfont{date}{\color{white}{\insertdate}}
    \end{beamercolorbox}
  \endgroup
  \vfill
}


\begin{document}

%Seteando el color del titulo del frame
\setbeamercolor{frametitle}{fg=white}

%********************PRIMERA DIAPOSITIVA********************
\usebackgroundtemplate{\includegraphics[width=\paperwidth, trim = -8.15cm 50.5cm 0 0]{fondo-primera-y-ultima-diapositiva.png}}
\begin{frame}
%Transición de un frame a otro
\transdissolve
\titlepage
\end{frame}

%*******************DIAPOSITIVAS RESTANTES********************
\usebackgroundtemplate{\includegraphics[width=\paperwidth, trim = -8.15cm 50.5cm 0 0]{fondo-resto-de-diapositivas.png}}
\begin{frame} {Tabla de contenido}
\transdissolve
\tableofcontents
\end{frame}

\begin{frame}{Mi primer frame}
\transdissolve
Mi texto de prueba
\\~
\alert{mi texto de alerta}
\end{frame}

%*******************ULTIMA DIAPOSITIVA********************
\usebackgroundtemplate{\includegraphics[width=\paperwidth, trim = -8.15cm 50.5cm 0 0]{fondo-primera-y-ultima-diapositiva.png}}
\begin{frame}{¡Gracias!}
%Transición de un frame a otro
\transdissolve
\titlepage
\end{frame}
\end{document}
